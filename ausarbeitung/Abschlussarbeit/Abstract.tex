\paragraph{}
Moderne Suchmaschinen bieten zur Zeit keine Übersicht über die gefundene Ergebnismenge, was die Suche für den Endbenutzer komplizierter macht, da alle gefundene Dokumente angeschaut werden müssen, um einen Überblick über die gefundene Daten zu bekommen, und um Anregungen für präzisere Anfragen zu gewinnen.

\paragraph{}
Um solche Benutzerunterstützung bei der Websuche zu ermöglichen, müssen Suchmaschinen den Inhalt von Webseiten auswerten können. Eine der solchen Auswertungen ist die Erkennung und Extraktion von sogenannten Entitäten. Entität kann etwa eine Person, ein Datum, oder eine Organisation sein. Eine Entität hat außerdem verschiedene Attributen, wie Klasse, Name, kurze Beschreibung und Verbindungen zu anderen Entitäten. Eine Anreicherung von Suchergebnissen mit den Informationen über extrahierte Entitäten wäre bei der Bearbeitung von Suchergebnissen und bei der Verfeinerung von Suchanfragen sehr hilfreich.

\paragraph{}
Im Rahmen dieser Arbeit muss ein Framework entwickelt werden, der:
\begin{itemize}
\item Die Suchanfrage des Endbenutzers an eine konventionelle Suchmaschine weiterleitet.
\item Erkennt die Entitäten auf von der Suchmaschine zurückgegebenen Suchergebnisseiten. 
\item Verlinkt die extrahierte Entitäten mit den Ontologien aus DBpedia.
\item Die Suchergebnisse mit den Ontologien anreichert, und an den Benutzer als ein XML oder JSON-Dokument schickt.
\end{itemize}