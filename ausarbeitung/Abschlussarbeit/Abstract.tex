Im Rahmen dieser Arbeit wird eine neue Möglichkeit von Benutzerunterstützung bei der Websuche erforscht, die dem Benutzer einen Übersicht über in den Suchergebnissen vorhandene Entitäten wie Personen, Firmen, Städte oder Länder, liefert. Es wird eine Erweiterung für das Stanbol-NLP-Framework entwickelt, die verschiedene NER\footnote{Named Entity Extraktion - Extraktion von Entitäten}-Algorithmen implementiert. Zusätzlich dazu wird eine API aufgebaut, die dem Entwickler, der keine Vorwissen über NLP\footnote{Natural Language Processing - Natürlichsprachliche Mensch-Computer-Interaktion} besitzt, die Möglichkeit gibt, die Extraktion von Entitäten in eine beliebige Suchapplikation zu integrieren. 

Die vorliegende Arbeit ist auf deutschsprachige Webseiten orientiert, aber die Grundprinzipien lassen sich auch auf andere Sprachen erweitern.

Anschließend durchgeführte Evaluierung zeigt, dass die Entitäten tatsächlich aus den kurzen Textsnippets, die auf den Suchergebnisseiten vorhanden sind, extrahiert werden können. Es wird auch gezeigt, dass die in der NLP üblich verwendete Metriken wie F-Measure nicht unbedingt für die Bewertung der Qualität von Extraktion von Entitäten verwendet werden können, da jeder Benutzer seine eigene Vorstellungen davon hat, welche von den aus dem Text extrahierten Entitäten ,,passend`` und welche ,,unpassend`` sind. Zusätzlich dazu wird die Wichtigkeit von den mit den Entitäten verlinkten Informationen (wie Geburtsort einer Person) gezeigt. Dadurch wird gezeigt, dass eine reine Extraktion von Entitäten nur ein Bruchteil der Aufgabe ist, und dass eine Verlinkung von Informationen über gefundene Entitäten eine wichtige Unteraufgabe ist.