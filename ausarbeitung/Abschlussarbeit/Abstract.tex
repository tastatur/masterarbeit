Im Rahmen dieser Arbeit wird eine neue Möglichkeit von Benutzerunterstützung bei der Websuche erforscht, die dem Benutzer eine Übersicht über in den Suchergebnissen vorhandene Entitäten wie Personen, Firmen, Städte oder Länder, liefert. Es wird eine Erweiterung für das Stanbol-NLP-Framework entwickelt, die verschiedene NER\footnote{Named Entity Extraktion - Extraktion von Entitäten}-Algorithmen implementiert. Zusätzlich dazu wird eine API aufgebaut, die dem Entwickler, der kein Vorwissen über NLP\footnote{Natural Language Processing - Natürlichsprachliche Mensch-Computer-Interaktion} besitzt, die Möglichkeit gibt, die Extraktion von Entitäten in eine beliebige Suchapplikation zu integrieren. 

Die vorliegende Arbeit ist auf deutschsprachige Webseiten orientiert, aber die Grundprinzipien lassen sich auch auf andere Sprachen erweitern.

Anschließend durchgeführte Evaluierung zeigt, dass das entwickelte System tatsächlich für die Benutzerunterstützung verwendet werden könnte, obwohl die graphische Oberfläche des Systems noch angepasst werden muss. Es wird außerdem gezeigt, dass es wichtig ist, dass die Suchergebnisse nicht mit allen gefundenen Entitäten angereichert werden, sondern nur mit den Entitäten, die als ,,wichtig`` eingestuft wurden. Als Maß für die ,,Wichtigkeit`` einer Entität werden in dieser Arbeit die Gewichte, die direkt von Extraktionsalgorithmen bestimmt werden, verwendet. Es wird aber gezeigt, dass solches Maß Nachteile hat, und dass die ,,Wichtigkeit`` einer Entität in den zukünftigen Arbeiten für jeden Benutzer personalisiert definiert werden soll.