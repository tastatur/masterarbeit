\chapter{Implementierung}
\label{sec:Implementierung}

\section{Apache stanbol}
Beschreibung von Stanbol. Schema aus der Präsentation. Was genau muss erweitert werden? Warum verwenden wir Stanbol überhaupt? Warum nicht AlchemyAPI? (der Grund ist - keine Anpassungsmöglichkeit/Quelltexte + das ist ein BlackBox - wir wissen nicht mal, wie genau die Entitäten dort extrahiert werden + Preis, denke ich).

\section{Extraktion von Entitäten}
Kurze Einleitung, wie genau die Annotationen für die Entitäten erzeugt werden, unabhängig von dem benutzten Einsatz.

\subsection{StanfordNER} 
%Beschreibung der Implementierung des Einsatzes von StanfordNER. 
\paragraph{}
Die erste Anreicherungskette wurde auf Basis von StanfordNER aufgebaut. Dieser Framework implementiert CRF-Einsatz zur Extraktion von Entitäten, und stellt ein Modell \footnote{\url{http://www.nlpado.de/~sebastian/software/ner_german.shtml}} für die deutsche Sprache zur Verfügung. Der Nachteil ist, dass es nur ein trainiertes Modell zur Verfügung stellt - das Korpus selbst, auf deren Basis man eigenes Modell trainieren könnte, stehen nicht zur Verfügung. Es stehen insgesamt zwei Modellen zur Verfügung:

\begin{enumerate}
\item HGC - Huge German Corpus-generalized classifier - dieses Modell wurde auf Texten aus Zeitungen trainiert.
\item deWac - dieser Klassifikator wurde auf Texten aus Internet trainiert.
\end{enumerate}

\paragraph{}
Dieses Framework ist auch für Vorverarbeitung - Zerlegung des Textes in einzelne Sätze und Zerlegung von Sätzen in einzelne Tokens - verantwortlich. Deswegen sind für den auf Stanford-NER basierten Engine nur zwei Engines notwendig:
\begin{enumerate}
\item Tika-Engine für Extraktion von rohen Textdaten aus HTML.
\item Spracherkennungsmodul, mit deren Hilfe sichergestellt wird, dass die Analyse nur für deutsche Sprache gestartet wird.
\end{enumerate}

\subsection{OpenNLP}
Beschreibung des OpenNLP-Einsatzes.

\subsection{Training von Modellen}
Wie genau werden die Modellen für OpenNLP trainiert?

\subsubsection{TIGER Korpus}
\paragraph{}
TIGER\footnote{\url{http://www.ims.uni-stuttgart.de/forschung/ressourcen/korpora/tiger.html}} Korpus wurde von dem Institut für Maschinelle Sprachverarbeitung auf der Basis von Zeitungen aufgebaut, und beinhaltet 50474 Sätzen. Außer markierten Entitäten beinhaltet dieser Korpus auch die Informationen über POS (Part Of Speech - ob das Wort ein Verb oder ein Substantiv ist), Lemma (Infinitiv für Verben oder Nominativ Singular für Substantive) und andere Informationen über die annotierte Tokens, wie Kasus oder Genus. Der Ausschnitt des Korpuses findet man in der Auflistung \ref{lst:TIGERBEISPIEL}.

\lstinputlisting[captionpos=b,label={lst:TIGERBEISPIEL},caption={Ausschnitt aus dem TIGER-Korpus. Die Informationen, die in dieser Arbeit nicht gebraucht werden, wurden wegen Platzmangel weggelassen.}]{Listings/tiger-example.txt}

\paragraph{}
Der Korpus ist wie folgt aufgebaut:
\begin{itemize}
\item Die Sätze werden mit einer leeren Zeile getrennt.
\item Jeder Token und seine Annotationen werden in einer Zeile geschrieben. Die Spalten werden mit einem Leerzeichen getrennt.
\item Die erste Spalte beinhaltet ID des Tokens, die aus Nummer des Satzes und Nummer des Tokens innerhalb des Satzes besteht.
\item Die zweite Spalte ist der Token selbst, so wie er auch im Satz vorkommt.
\item In der dritten Spalte steht Lemma des Tokens.
\item Die vierte Spalte wird nicht verwendet.
\item Und die fünfte Spalte beinhaltet POS-Tag des Tokens. Falls dieser Tag den Typ \textit{NE} hat, ist das eine Entität.
\end{itemize}

Der Nachteil dieses Korpuses besteht darin, dass es zwischen verschiedenen Typen von Entitäten nicht unterschieden wird, und die alle den Typ \textit{NE} haben.

\paragraph{}
Um ein MaximumEntropy-Modell mithilfe von OpenNLP auf Basis von TIGER-Korpus aufbauen zu können, muss der Korpus zuerst ins OpenNLP-Format umgewandelt werden. Der Ausschnitt aus dem TIGER-Korpus im OpenNLP-Format findet man in der Auflistung \ref{lst:TIGEROPENBEISPIEL}.

\lstinputlisting[captionpos=b,label={lst:TIGEROPENBEISPIEL},caption={Ausschnitt aus dem TIGER-Korpus im OpenNLP-Format.}]{Listings/tiger-opennlp.txt}

\subsubsection{Wikipedia-basiertes Korpus}
\paragraph{}
Leider kann man nicht immer einen manuell aufgebauten Korpus verwenden, entweder aus Lizenz- oder Kostengründen. Viele Korpusse sind nur für Forschung frei verfügbar, was bedeutet, dass 
die auf keinen Fall in einem Geschäftsprojekt verwendet werden dürfen. Und einen eigenen Korpus aufzubauen ist auch nicht immer möglich, da dazu die Linguisten eingesetzt werden müssen, die nicht jede Firma zur Verfügung hat, und es kann Monaten dauern, bis man ein eigenes Korpus erstellt. Was könnte man in diesem Fall tun?

\paragraph{}
Oliver Grisel\footnote{\url{http://www.nuxeo.com/blog/mining-wikipedia-with-hadoop-and-pig-for-natural-language-processing/}} hat einen interessanten Einsatz zur automatischer Aufbau von Korpus vorgeschlagen. Es wird vorgeschlagen, Wikipedia als Textbasis zu nehmen, und die interne Links, die auf andere Wikipediaseiten führen, sollen als  Entitäten markiert werden. Es soll anschließlich ein Korpus aufgebaut werden, auf dessen Basis ein OpenNP-Modell trainiert werden soll. Da Wikipedia auch auf deutscher Sprache verfügbar ist, soll dieser Einsatz auch für Zwecke dieser Arbeit nützlich sein.

\paragraph{}
Um den Korpus aufzubauen, wird Apache Pig verwendet - ein Framework für Big-Data-Analyse, der eine Skript-Sprache und JavaAPI zur Verfügung stellt. Die Aufbau vom Korpus umfasst folgende Schritte:
\begin{enumerate}
\item Es wird Dump von Wikipediaartikeln heruntergeladen.
\item Es werden die Listen von Wikipedia-Links und Typen von Entitäten von DBpedia heruntergeladen. Diese Informationen braucht man später, um jeder Entität in dem erzeugten Korpus die richtige Entitätstyp zuordnen zu können.
\item Es werden Links auf interne Wikipedia-Artikeln aus Wikipedia-Dump extrahiert, mit der Positionsinformation zusammen.
\item Jeder Link wird mithilfe von DBpedia-Daten einen Entitätstyp zugeordnet.
\item Es wird ein Trainingskorpus im OpenNLP-Format erzeugt.
\end{enumerate}

\paragraph{} 
Für Korpusaufbau aus deutscher Wikipedia kann im Prinzip die Scripts von Oliver Grisel genommen werden, die müssen allerdings angepasst werden:
\begin{itemize}
\item Es soll deutsche Dbpedia, und nicht englische verwendet werden.
\item Es soll deutsches Modell für Satzerkennung anstatt englisches eingesetzt werden.
\item Herunterladen von Wikipedia- und Dbpediadaten soll automatisiert werden.
\end{itemize}

\paragraph{}
Aber welche Vor- und Nachteile hat automatische Erzeugung vom Korpus? Kann das trainierte Modell später auch tatsächlich für sinnvolle Entitätserkennung eingesetzt werden?
\begin{itemize}
\item Vorteile
\begin{itemize}
\item Die Erzeugung von Korpus braucht höchstens eine Stunde, im Vergleich zu manuell annotierten Korpussen.
\item Es werden keine Fachleute gebraucht, um Korpus zu erzeugen.
\end{itemize}
\item Nachteile
\begin{itemize}
\item Nicht alle interne Links stellen eine Entität dar, und nicht alle Entitäten sind ein Link - als Folge ist die Qualität des Korpusses deutlich niedriger, als die von manuell aufgebauten.
\item Eine sehr niedrige Varianz - alle Wikipediaartikel sind mehr oder weniger im gleichen Styl geschrieben, was bedeutet, dass wenn man dem trainierten Modell einen Text zeigt, der sich von einem durchschnittlichen Wikipedia-Artikel deutlich unterscheidet, werden da höchstwahrscheinlich keine Entitäten gefunden.
\end{itemize}
\end{itemize}

\subsection{Zwischenfazit}
Vergleich aller Ansätze.

\section{Verlinkung und Dereferinzierung von annotierten Entitäten}
\paragraph{}
Nachdem die Entitätserkennung durchgeführt wurde, hat man die Informationen darüber, ob es Entitäten im Text gefunden wurden, die Position der Entität innerhalb des Satzes und optional den Typ der Entität. Die Ontologie selbst, die Endbenutzer zu sehen braucht, fehlt allerdings noch. Es müssen noch zwei Schritte durchgeführt werden, bis die Informationen komplett sind - Verlinkung von gefundenen Entitäten mit der Entitäten in DBpedia (oder einer anderer Wissendatenbank) und die Dereferinzierung von Eigenschaften der Entität.

\paragraph{}  
Während der Verlinkung von Entitäten wird für jede erkannte Entität eine Suche nach dieser Entität in einer oder mehreren Wissendatenbanken durchgeführt. Um den Vorgang höchstmöglich zu beschleunigen, soll lokaler Wissendatenbankindex (EntityHub in Terminologie von Stanbol) eingesetzt werden. Dieser Index soll vorher aufgebaut werden, und zumindest die Namen der Entitäten beinhalten. Für spätere Schritte ist aber empfehlenswert, auch diverse Eigenschaften von Entitäten zum Index hinzuzufügen, damit Dereferinzierungschritt auch so schnell wie möglich ausgeführt wird. Es muss aber beachtet werden, dass Index von realen Wissendatenbanken wie DBpedia mehr als 20 Gigabytes wiegen kann, und die Aufbau kann mehrere Tagen in Anspruch nehmen.

\paragraph{}
Für die Verlinkung von Entitäten wird Engine NamedEntityTaggingEngine verwendet. EntityHub stellt dabei eine Suchschnitstelle zur Verfügung. Alle im Index gefundene Einträge werden nach Levenshtein-Distanz zum Text der Entität und nach Rank der Entität innerhalb EntityHub sortiert.


\paragraph{}
Als Wissendatenbank für die Verlinkung und Dereferinzierung wird in dieser Arbeit lokaler Index von deutscher DBpedia verwendet.

\section{API für Anreicherung von Suchergebnissen}
\paragraph{}
Um dem Endentwickler die Anreicherung von Suchergebnissen so einfach wie möglich zu machen, wurde eine API entwickelt, die man direkt an sein Projekt als eine Bibliothek anbinden kann. Diese Bibliothek wird auch später in der Evaluierung verwendet. Es wird eine Abstraktionsschicht hinzugefügt, die die Aufrufe der REST-Schnittstelle von Stanbol hinter der Klientklasse verbirgt. Der Entwickler soll nur die Liste von Suchsnippets an API übergeben, für die Verbindungaufbau zum Stanbol und Parsing der Antwort des Servers ist API verantwortlich. Die Beschreibung der API-Schnittstelle findet man in der Auflistung \ref{lst:APISCHNITSTELLE}. 

\lstset{language=Java}
\lstinputlisting[captionpos=b,label={lst:APISCHNITSTELLE},caption={Die API-Schnittstelle zum Stanbol}]{Listings/api.txt}

Als Eingabedaten übergibt man die Liste von URLs gefundenen Webseiten mit dazugehörigen Texten von Snippets zusammen, und als Ausgabe bekommt man für jede URL die Liste von gefundenen Entitäten. Da in Rahmen dieser Arbeit mehr verschiedenen Engineketten implementiert wurden, muss der Name der erwünschten Kette miteingegeben werden.