\chapter{Implementierung}
\label{sec:Implementierung}

\section{Apache stanbol}
Beschreibung von Stanbol. Schema aus der Präsentation. Was genau muss erweitert werden? Warum verwenden wir Stanbol überhaupt? Warum nicht AlchemyAPI? (der Grund ist - keine Anpassungsmöglichkeit/Quelltexte + das ist ein BlackBox - wir wissen nicht mal, wie genau die Entitäten dort extrahiert werden + Preis, denke ich).

\section{Extraktion von Entitäten}
Kurze Einleitung, wie genau die Annotationen für die Entitäten erzeugt werden, unabhängig von dem benutzten Einsatz.

\subsection{StanfordNER} 
Beschreibung der Implementierung des Einsatzes von StanfordNER. 

\subsection{OpenNLP}
Beschreibung des OpenNLP-Einsatzes.

\subsection{Training von Modellen}
Wie genau werden die Modellen für OpenNLP trainiert?

\subsubsection{TIGER Korpus}
\paragraph{}
TIGER\footnote{\url{http://www.ims.uni-stuttgart.de/forschung/ressourcen/korpora/tiger.html}} Korpus wurde von dem Institut für Maschinelle Sprachverarbeitung auf der Basis von Zeitungen aufgebaut, und beinhaltet 50474 Sätzen. Außer markierten Entitäten beinhaltet dieser Korpus auch die Informationen über POS (Part Of Speech - ob das Wort ein Verb oder ein Substantiv ist), Lemma (Infinitiv für Verben oder Nominativ Singular für Substantive) und andere Informationen über die annotierte Tokens, wie Kasus oder Genus.

\paragraph{}


\subsubsection{Wikipedia-basiertes Korpus}
Wie kann Wikipedia als Korpus verwendet werden? Vor- und Nachteile im Bezug auf manuell aufgebaute Korpusse.

\subsection{Zwischenfazit}
Vergleich beider Ansätze.

\section{Verlinkung von annotierten Entitäten}
Was ist eine Verlinkung? Wie genau funktioniert die? Aufbau von EntityHub. Indexierung von deutscher DBpedia.

\section{Proxierung von Benutzersuchanfragen}
Wie genau ist der ProxyServer aufgabaut (@Sebastian, haben wir den Zugang zu den Quelltexten von Proxy, oder zumindest die Login/Password-Daten für Bing-API?)? Schema zeichnen, wie in der Beschreibung von Stanbol. Einleitung in BingAPI, Preis/Leistung-Vergleiche von Google/Bing/Yahoo-API.